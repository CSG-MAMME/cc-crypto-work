\section{Problem 2: Simple PRF from DDH}\label{sec:problem2}

Let $\mathbb{G}$ be a cyclic group of prime order $q$ generated by $g \in \mathbb{G}$.
Let $H : \mathcal{M} \rightarrow \mathbb{G}$ be a hash function, which we shall model as a random oracle.
Let $F$ be the PRF defined over $(\mathbb{Z}_{q},\mathcal{M},\mathbb{G})$ as follows:
\begin{equation*}
    F(k,m) \coloneqq H(m)^{k} \hspace{8pt} \text{for } k \in \mathbb{Z}_{q}, m \in \mathcal{M}.
\end{equation*}
Show that $F$ is a secure PRF in the random oracle model for $H$ under the DDH assumption for $\mathbb{G}$.
In particular, you should show that for every adversary $\mathcal{A}$ attacking $F$ as a PRF, there exists a DDH adversary $\mathcal{B}$, which is an elementary wrapper around $\mathcal{A}$, such that $PRFadv[\mathcal{A}, F] \leq DDHadv[\mathcal{B}, \mathbb{G}] + \frac{1}{q}$.

\begin{center}
    \rule{5cm}{0.4pt}
\end{center}

\textbf{\textit{Proof:}}
To solve this exercise we will see that if we can break the PRF game then the two distributions of the exercise 10.10 are not computationally indistinguishable.
Let's recall what the PRF game is about:

For a given PRF $F$, defined over $(\mathcal{K}, \mathcal{X} , \mathcal{Y})$, and for a given adversary
A, we define two experiments, Experiment $0$ and Experiment $1$. For $b=0,1$, we define:

Experiment $b$:
\begin{itemize}
\item 
The challenger selects $f \in Funs[\mathcal{X},\mathcal{Y}]$ as follows:

if $b = 0: k\leftarrow K, f\leftarrow F(k, ·)$;
if $b = 1: f\leftarrow Funs[X , Y]$.
\item 
The adversary submits a sequence of queries to the challenger.
For $i = 1, 2,\ldots,$ the ith query is an input data block $x_{i}\in\mathcal{X}$ .
The challenger computes $y_{i}\leftarrow f(x_{i})\in \mathcal{Y}$, and gives $y_{i}$ to the adversary.
\item
The adversary computes and outputs a bit $\hat{b}\in {0,1}$.
\end{itemize}

So we consider the following triplet:
$$
(g^{x},g^{k},T)
$$
Where $g^{x}=H(m)$ for some message, $g^{k}$ is an element of the group $g\in\mathbb{G}$  and $k$ the element chosed randomly in the PRF game. And $T$ the final output of the PRF game. So, if the challenger does experiment $0$ then $T=g^{xk}$, and if does experiment $1$ $T$ is random element of $\mathbb{G}$.

So, as can break the game of the pseudorandom function i can disting between $T=g^{kx}$ and $T$ with some no negligible probability as a random element. So if i consider the tripets:
\begin{equation*}
    (g^{x},g^{k},g^{xk}), \hspace{5pt} (g^{x},g^{k},T)
\end{equation*}
When the challenger makes experiment $0$ it will correspond to the triplet $(g^{x},g^{k},g^{xk})$ and when he plays 

Now as I can break the game I can distinguish between the two triplets and this is impossible by the DDH assumption, so I arrive to a contradiction supposing that I can break the PRF game.

$$PRFadv[\mathcal{A},F]=|Pr(W_{0})-Pr(W_{1})|$$

where $W_{b}$ is the event that the adversary outputs $b$ in experiment $b$. In our case we have that:

$$PRFadv[\mathcal{A},F]=Distadv[\mathcal{A},\mathcal{D},\mathcal{R}]$$

because in this case distinguish between the two experiments is the same of distinguish between the two tripets so by exercise 10.10:

$$PRFadv[\mathcal{A},F]=Distadv[\mathcal{A},\mathcal{D},\mathcal{R}]\leq \frac{1}{q}+ DDHadv[\mathcal{B},\mathbb{G}]$$

